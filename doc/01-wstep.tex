\chapter{Wstęp}
Proces informatyzacji przedsiębiorstw, rozpoczęty kilka dekad temu, wprowadził światową gospodarkę na~nowe, dotąd nieznane tory rozwoju. Skrócenie procesu produkcyjnego, stosowanie kontroli komputerowych, czy~też~skomputeryzowanych maszyn przyspieszyło i~ułatwiło produkcję,~a~także zarządzanie procesami w~firmach i~przedsiębiorstwach. Przed ludźmi stanęły możliwości, ale także wyzwania, z~którymi nigdy wcześniej nie~musieli sobie radzić. Zmiany, jakie nastąpiły przez ostatnie trzy dekady są nieodwracalne i~zmuszają programistów do~tworzenia nowych aplikacji, które będą w~stanie sprostać stawianym im~wymaganiom.

Informatyzacja firm, instytucji oraz innych jednostek organizacyjnych powinna realizować dwa podstawowe cele. Z~jednej strony powinna ona usprawniać pracę pojedynczego pracownika poprzez automatyzację realizowanych przez niego rutynowych zadań. Dzięki wykorzystaniu możliwości komputerów działania te~powinny być wykonywane szybciej i~w~sposób bardziej niezawodny. Z~drugiej strony celem informatyzacji jest wpływanie na~działanie całych firm w~wyniku wspomagania decyzji kadry zarządzającej przedsiębiorstwami. Szybka analiza bazująca na~pełnej i~aktualnej informacji o~stanie firmy może ułatwić kadrze zarządzającej podejmowanie trafnych i~odpowiednio wczesnych decyzji o~strategicznym znaczeniu dla~rozwoju danego przedsiębiorstwa.

Wprowadzenie komputerów do~właściwie każdej przestrzeni ludzkiego życia wpłynęło na~wyprodukowanie olbrzymich ilości danych. Reprezentowane są one w~sposób umożliwiający ich składowanie i~przetwarzanie komputerowe przez aplikacje analityczne. W~chwili obecnej ludzkość jest świadkiem eksplozji w~produkcji danych tworzonych przez różnego rodzaju systemy komputerowe. Analiza tych danych przynieść może wymierne korzyści nie~tylko w~kwestiach finansowych, ale~również poznawczych. Dzięki analizie zebranych w~przeszłości informacji możliwe jest lepsze dopasowanie planów w~przyszłości - na~tej podstawie planowane mogą być np. akcje marketingowe, czy~też~promocje w~supermarketach spożywczych. Wykorzystanie wiedzy uzyskanej w~ten sposób jest niezwykle szerokie i~może być użyte w~każdym obszarze działalności firmy.

Odkrycie zależności pomiędzy zgromadzonymi danymi bez zastosowania narzędzi informatycznych jest procesem bardzo skomplikowanym i~wymagającym dużo czasu do~realizacji. Przy obecnej złożoności większości systemów oraz rozmiarom danych produkowanych przez te systemy, koszt czasowy jest na~tyle duży, że~ręczna analiza tych danych stała się~niemożliwa. Dlatego też tworzone są narzędzia umożliwiające odkrywanie prawidłowości w~dużych zbiorach danych, by człowiek na~tej podstawie mógł podejmować decyzje i~wyciągać wnioski.

Dział informatyki, który~zajmuje się~odkrywaniem ukrytych dla~ludzi prawidłowości i~reguł w~danych nazywa się~eksploracją danych (\english{Data Mining}, w~literaturze spotkać można również określenie drążenie danych, ekstrakcja danych, pozyskiwanie wiedzy, czy~też~wydobywanie danych~\cite{Elmasri:db}), który~jest jednym z~etapów procesu \termdef{odkrywania wiedzy z~baz danych} (\acronym{KDD}, \english{Knowledge Discovery in Databases}). Proces odkrywania wiedzy w~bazach danych obejmują zwykle działania bardziej złożone niż tylko eksploracja danych. Są to~między innymi selekcja danych, transformacja lub~kodowanie danych, czy~też~raportowanie i~prezentowanie odkrytych informacji~\cite{Elmasri:db}. Eksploracja danych to~proces odkrywania wiedzy w~postaci nowych, użytecznych, poprawnych i~zrozumiałych wzorców w~bardzo dużych wolumenach danych~\cite{DataMiningStart}. Możliwości stosowania technik eksploracji danych w~praktyce, wymagają efektywnych metod przeszukiwania ogromnych plików lub baz danych. Warto przy~tym~wspomnieć, że~tego typu technologie nie~są w~chwili obecnej dobrze zintegrowane z~systemami zarządzania bazami danych.

Eksploracja danych odbywa się~najczęściej w~środowisku baz lub~hurtowni danych, które~stanowią doskonałe źródła danych do~analizy - głównie ze względu na~łatwość dostępu oraz usystematyzowaną strukturę przechowywanych informacji. Ponieważ liczba odkrytych wzorców w~wielu przypadkach może być bardzo duża, odkryte wzorce bardzo często zapisuje się~w osobnych relacjach bazy lub hurtowni danych. Pozwala to~na ich dalsze przetwarzanie w~trybie off-line przez użytkowników końcowych. Pojęcie eksploracji zyskuje coraz większą popularność (również w~wymiarze marketingowym) i~jest wykorzystywane w~wielu dziedzinach ludzkiego życia.

Jednym z~najczęściej wykorzystywanych modeli wiedzy w~eksploracji danych są reguły asocjacyjne. Reguła asocjacyjna ma postać $X \Rightarrow Y$, gdzie $X$ oraz $Y$ są wzajemnie rozłącznymi zbiorami elementów. Przykładem reguły, która mogła zostać odkryta w~bazie danych sklepu komputerowego, może być reguła postaci $komputer \land myszka \Rightarrow monitor$. Prezentuje ona fakt, że~klienci kupujący komputer oraz myszke z~dużym prawdopodobieństwem kupią również monitor. W~\cite{Problem:Statement} po raz pierwszy sformułowany został problem odkrywania reguł asocjacyjnych wraz z~algorytmem Apriori, który~jest podstawą wielu algorytmów znajdujących reguły asocjacyjne. Algorytm ten~został następnie rozszerzony w~pracy~\cite{AssRulesStrt}.

W ostatnich latach pojawiły się~nowe możliwości wykorzystania współczesnych komputerów. W~roku 2007 firma NVIDIA udostępniła programistom uniwersalną architekturę obliczeniową CUDA (\english{Compute Unified Device Architecture}), który~umożliwia wykorzystanie mocy obliczeniowej \termdef{procesorów graficznych} (\acronym{GPU}, \english{Graphics Processing Unit}), bądź innych procesorów wielordzeniowych, do~rozwiązywania ogólnych problemów obliczeniowych w~sposób znacząco wydajniejszy niż w~przypadku tradycyjnych, sekwencyjnych procesorów~\cite{cuda:zone}. Choć w~grach komputerowych moc obliczeniową jednostek graficznych można wykorzystać do~obliczeń fizyki, to~CUDA idzie jeszcze dalej, umożliwiając przyspieszenie obliczeń w~takich dziedzinach, jak biologia, fizyka, kryptografia, bioinformatyka oraz innych naukach. Specjalnie dla~potrzeb tego segmentu NVidia opracowała kartę graficzną o~nazwie \emph{Tesla}~\cite{cuda:tesla}. Układy te są pierwszymi układami produkowanymi na~masową skalę, które przeznaczone zostały do~pracy \termdef{obliczeniach ogólnego przeznaczenia na~układach GPU} (\acronym{GPGPU}, \english{General-Purpose Computing on Graphics Processing Units}), czyli segmencie do~tej pory zarezerwowanym dla~klasycznych procesorów obliczeniowych.

Dotychczas bardzo małe jest zainteresowanie wykorzystaniem tej~technologii w~procesie odkrywania wiedzy, a~w~szczególności znajdowania reguł asocjacyjnych. Wyniki przeprowadzonych eksperymentów pozwalają przypuszczać, że~algorytm wykorzystujący możliwości procesorów wielordzeniowych, a~w~szczególności GPU, będzie wyraźnie szybszy od~klasycznych algorytmów eksploracji danych zaimplementowany na~tradycyjnych procesorach.

\section{Cel i~zakres pracy}
Celem pracy jest zaprojektowanie i~zaimplementowanie algorytmu odkrywającego reguły asocjacyjne, który~będzie wykorzystywał możliwośći współczesnych kart graficznych dzięki wykorzystaniu technologii CUDA oraz porównanie zaprojektowanego i~zaimplementowanego algorytmu do~innych, podstawowych algorytmów odkrywania reguł asocjacyjnych. W~ramach pracy dokonane zostanie również zebranie wiedzy dotyczącej algorytmów eksploracji reguł asocjacyjnych.

Niniejszy rozdział jest wprowadzeniem do~pracy, w~której opracowany i~zaimplementowany został równoległy algorytm odkrywający reguły asocjacyjne. W~rozdziale~\ref{chap:teoria} zebrano aktualną wiedzę na~temat algorytmów oraz zagadnienia związane z~odkrywaniem reguł asocjacyjnych - w~tym dwa podstawowe algorytmy, jakimi są Apriori oraz FPGrowth. Kolejna ważna część pracy, to~technologia CUDA, którą opisano w~rozdziale~\ref{chap:cuda}. Pozostałe użyte w~pracy technologie przedstawiono w~rozdziale~\ref{chap:technologie}. Rozdział~\ref{chap:algorytm} zawiera opis równoległego algorytmu korzystającego z~możliwości kart graficznych. Wyniki z~wykonanych eksperymentów zostały opracowane i~zebrane w~rozdziale~\ref{chap:eksperymenty}. Podsumowanie pracy zawarte jest w~rozdziale~\ref{chap:zakonczenie}.