\chapter{Technologie}\label{chap:technologie}

Niniejszy rozdział zawiera opis technologii użytych w~procesie realizacji projektu. Przedstawiono najważniejsze cechy każdej z~nich. Dokonano również podstawowego wprowadzenia do każdej z~wymienionych technologii, co jest potrzebne do pełnego przedstawienia sposobu ich wykorzystania.

\section{Język C\# i platforma .NET}\label{sec:cs}
Język C\# oraz .NET zostały pierwszy raz przedstawione podczas międzynarodowej konferencji dla programistów o nazwie \emph{Professional Developers Conderence} (\acronym{PDC}) w lipcu roku 2000~\cite{ms:initDotNet} - choć środowisko programistów słyszło o planach firmy Microsoft znacznie wcześniej, jednakże pod różnymi nazwami (m.in. COOL, COM3, Lightning). Pomimo sporego w owym czasie szumu wokół niedoskonałych systemów operacyjnych (np. Windows Me), czy też sporych opóźnień w dostarczaniu owych rozwiązań, firma zdecydowała się na stworzenie własnej platformy programistycznej~\cite{cSharp:inDepthF}.

Należy również podkreślić, że firma Microsoft nie była jedyną, która wspierała tworzenie języka - oprócz niej w sponsoring tego języka włączył się firmy Helwett-Packard oraz Intel. Język C\# jest, podobnie, jak język Java, otwartym językiem, jednakże w zupełnie innym tego słowa znaczeniu. Microsoft sprzyja ścieżce standaryzacyjnej języka, gdy Sun (firma posiadająca prawa do platformy Java) stopniowo udostępnia kod źródłowy Javy na zasadach \emph{wolnego oprogramowania} (\english{open source}) i dopuszczając, a nawet zachęcając do tworzenia innych \emph{środowisk uruchomieniowych} (\english{runtime environments}) Javy. Istnieją alternatywne implementacje CLI oraz C\# (najbardziej znanym jest Mono~\cite{cs:mono}), jednakże należy podkreślić, że nie obejmują one swoją implementacją wszystkiego, co kryje się pod nazwą Microsoft .NET Framework. Ostatecznie język C\# oraz .NET zostały udostępnione w roku 2002, razem z narzędziem programistycznym Vistual Studio .NET 2002.

\subsection{Platforma .NET}
Kiedy wprowadzono termin ''.NET'', został on wcielony w szereg różnych technologii wychodzących spod sztandarów firmy Micorosft. Na przykład Windows Live ID był nazywany .NET Passport, pomimo tego, że nie istniało żadne bezpoźrednie połączenie pomiędzy tą technologią, a tym co jest obecnie znane, jako .NET. Na całe szczeście z czasem zaprzestano używanie .NET, który w chwili obecnej wiąże się jedynie ze światem programowanie.

Poniżej przedstawione zostaną trzy składniki, z jakich zbudowana jest platforma .NET oraz łączące je zależności.

\subsubsection{Język, środowisko uruchomieniowe i biblioteki}
W skład platformy .NET wchodzą trzy podstawowe składniki: język programowania, biblioteki oraz środowisko uruchomieniowe. Wprawdzie rozróżnienie tych trzech nie zawsze jest możliwe, to ważna jest świadomość istnienia wyraźnych różnic pomiędzy tymi trzema składnikami.

\paragraph{Język}
Język C\# jest zdefiniowany przez swoją specyfikację, która opisuje format kodu programu w C\#, obejmując jednocześnie składnię oraz zachowanie. Nie jest tam zawarta natomiast informacja o platformie, na jakiej zostanie on uruchomiony, poza kilkoma kilkoma aspektami współpracy kompilatora z platformą. 

W teorii dowolna platforma, która wspiera wymagane funkcje (opisane w specyfikacji), może posiadać kompilator, który będzie budował oprogramowanie przeznaczone dla danej platformy.  Na przykład, kompilator języka C\# mógłby produkować na podstawie kodu źródłowego dowolny inny format niż tzw. \emph{język pośredni} (\acronym{IL}, \english{Intermediate Language}).

\paragraph{Środowisko uruchomieniowe}
Środowisko uruchomieniowe jest częścią .NET odpowiedzialną za dokonanie wszystkiego, by uruchomiony kod IL działał zgodnie ze specyfikacją jęzka. W implementacji firmy Microsoft srodowisko uruchomieniowe nazywa się \emph{Common Language Runtime}, czyli w skrócie \acronym{CLR}.

\paragraph{Biblioteki}
\emph{Biblioteki} (\english{libraries}) dostarczają kod, który może być wykorzystywany przez oprogramowanie. Większość bibliotek frameworka .NET są gotowymi produktami języka IL (patrz punkt~\ref{sec:msil}) z \emph{kodem natywnym} (\english{native code}) wykorzystywanym tylko wtedy, gdy jest to potrzbne. Należy podkreślić, że kodu bilbiotek jest znacznie więcej niż kodu środowiska uruchomieniowego. W taki sam sposób można spojrzeć na samochód (biblioteki), który jest znacznie bardziej skomplikowaną jednostką niż sam jego silnik (środowisko uruchomieniowe).

Warto podkreślić, że istnieje podział na biblioteki standardowe oraz pozostałe, które nie zostały ujęte w specyfikacji języka. Pisząc program, który wykorzystuje tylko te pierwsze można mieć dużą pewność, że będzie on możliwy do uruchomienia na dowolnej implementacji platrofmy - czy to Mono, .NET, czy jakiejkolwiek innej.

Termin \emph{.NET} odnosi się do kombinacji środowiska uruchomieniowego wraz z bibliotekami dostarczanymi przez Microsoft, a także kompilatory C\# oraz VB.NET. .NET moze być pojmowany, jako cała \emph{platforma programistyczna} (\english{development platform}) zbudowana nad systemem Windows.


\subsection{Kompilacja i język MSIL}\label{sec:msil}

\subsection{Język programowania C\#}
\definicja{Java} jest obiektowym językiem programowania, zaprojektowanym i~obecnie rozwijanym prze \emph{Sun Microsystems} jako podstawowy komponent platformy Java. Przeznaczony jest do tworzenia kodu źródłowego aplikacji, który następnie przy pomocy kompilatora jest przekształcany do kodu bajtowego, czyli postaci która może być wykonana przez \definicja{wirtualną maszynę Java} (\akronim{JVM}, \english{Java Virtual Machine}). Tak przygotowany kod zawiera instrukcję w~języku maszynowym JVM, które są niezależne od rodzaju procesora oraz systemu operacyjnego, na którym kod został stworzony.

%\begin{figure}[h]
%\centering\includegraphics[width=7cm]{figures/roz6/helloWorld.png}
%\caption{Proces budowania i~uruchamiania oprogramowania w~języku Java}
%\label{rys:helloWorld}
%\end{figure}

\paragraph{Obiektowość}
\definicja{Programowanie obiektowe} (\akronim{OOP}, \english{Object Oriented Programming}) jest to paradygmat programowania, w~którym program definiuje się przy pomocy obiektów.
Podstawowe cechy języka, który realizuje paradygmat programowania obiektowego przedstawił Alan Kay w~odniesieniu do języka \definicja{Smalltalk}, pierwszego poprawnie zrealizowanego języka obiektowego, a~zarazem jednego z~poprzedników C\#. Cechy te opisują czyste podejście obiektowe.
\begin{description}
	\item \emph{Wszystko jest obiektem.} Obiekt można przedstawić jako specjalną zmienną, która nie tylko zawiera dane, ale i~także może realizować żądania, czyli wykonywać na swoich danych pewne ściśle określone operacje. Teoretycznie każdy element świata rzeczywistego np.: samochód, dom, zamek, pracownik może być reprezentowany w~programie przy pomocy tak skonstruowanego obiektu.
	\item \emph{Aplikacja jest zbiorem komunikujących się między sobą obiektów.} Przesyłanie komunikatów do obiektu to żądanie od niego realizacji pewnej operacji. Można to nazwać wykonaniem funkcji należącej do konkretnego obiektu. Przykładem może być następujący scenariusz: obiekt \emph{Kierowca} wysyła do obiektu \emph{Samochód} komunikat \emph{przyspiesz} co powoduję wykonanie operacji zwiększenia prędkości.
	\item \emph{Każdy obiekt posiada własną pamięć, na którą składają się inne obiekty.} Tworzenie nowego obiektu polega na łączeniu w~jeden element grupy już istniejących obiektów. Powstaje w~ten sposób wielowarstwowa aplikacja, która jednocześnie ukrywa swoją złożoność za prostymi obiektami. Przykładowo obiekt \emph{Samochód} można zbudować z~następujących obiektów \emph{Karoseria}, \emph{Koło}, \emph{Silnik} itd. Jednocześnie chcąc zwiększyć prędkość pojazdu odwołujemy się do obiektu \emph{Samochód}, a~nie bezpośrednio do obiektu \emph{Silnik}, który odpowiada za jego prędkość.
	\item \emph{Każdy obiekt posiada swój typ.} W~odniesieniu do obiektu słowo \emph{typ} można zastąpić słowem \definicja{klasa}. Każdy obiekt jest instancją pewnej klasy, której głównym zadaniem jest zdefiniowanie jakie komunikaty można wysłać do obiektu będącego jej egzemplarzem.
	\item \emph{Wszystkie obiekty danego typu obsługują te same komunikaty.} Każdy obiekt danej klasy obsługuje wszystkie komunikaty zdefiniowane w~klasie. Dodatkowo jeśli np. obiekt typu \emph{Student} jest jednocześnie obiektem typu \emph{Człowiek}, to obsługuje wszystkie komunikaty zdefiniowane dla typu \emph{Człowiek}. Umożliwia to pisanie bardziej uniwersalnego kodu, który będzie obsługiwał wszystkie obiekty pasujące do typu \emph{Człowiek}~\cite{cSharp:progr}.
\end{description}

Współcześnie, aby język programowania został uznany za obiektowy musi charakteryzować się wymienionymi poniżej cechami.
\begin{description}
	\item[Abstrakcja] Każdy obiekt systemu jest widziany jako abstrakcyjny ,,wykonawca'', który może realizować pracę, określać i~zmieniać swój stan oraz wysyłać komunikaty do innych obiektów w~systemie. Jednocześnie nie ujawnia on jak zostały zaimplementowane dane cechy.
	\item[Hermetyzacja] Czyli ukrywanie wewnętrznej implementacji obiektu. Gwarantuje to, że obiekt nie ma możliwości zmiany stanu wewnętrznego innego obiektu w~nieprzewidziany przez programistę sposób. Tylko metody (funkcje) składowe obiektu mają prawo do modyfikacji jego stanu. Dodatkowo pozwala to na swobodną modyfikację kodu przez programistę o~ile nie zmieniają się metody składowe. Przykładowo w~pierwszej wersji klasy programista może zaimplementować listę jako tablicę o~stałym rozmiarze, natomiast w~kolejnej wersji zmienić tablicę na listę jednokierunkową.
	\item[Polimorfizm] Referencje mogą wskazywać na obiekty różnego, ale zgodnego typu. Wywołanie metody dla referencji spowoduje wykonanie operacji odpowiedniej dla pełnego typu obiektu wywoływanego. Jeśli ma to miejsce w~trakcie działania programu, nazywa się to późnym wiązaniem lub wiązaniem dynamicznym.
	\item[Dziedziczenie] Umożliwia definiowanie specjalizowanych obiektów na podstawie ich ogólniejszych odpowiedników. Podczas definiowania obiektów specjalizowanych nie jest wymagana redefinicja całej funkcjonalności obiektu bazowego, ale tylko ta, której brakuje w~obiekcie ogólniejszym, lub ta, której sposób działania chcemy zmienić.
	
\end{description}

\section{CUDA}

\subsection{CUDA na platformie .NET}

\section{Środowisko programistyczne}

\subsection{Zarządzanie zasobami projektu}

\subsection{Budowanie projektu}

\section{Testowanie}

\subsection{Usprawnienie procesu testowania}
